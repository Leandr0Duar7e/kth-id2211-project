\documentclass{article}

% Use NeurIPS style
\usepackage[final]{neurips_2024}

% Additional packages
\usepackage[utf8]{inputenc}
\usepackage[T1]{fontenc}
\usepackage{hyperref}
\usepackage{url}
\usepackage{booktabs}
\usepackage{amsfonts}
\usepackage{nicefrac}
\usepackage{microtype}
\usepackage{xcolor}
\usepackage{graphicx}
\usepackage{algorithm}
\usepackage{algorithmic}
\usepackage{amsmath}
\usepackage{cite}
\usepackage{subcaption}
\usepackage{lineno}

% Activate line numbers
\linenumbers

\title{Temporal Evolution of Political Communities on Reddit: A Graph-Based Analysis}

\author{
  Andrei Iliescu \\
  KTH Royal Institute of Technology\\
  \texttt{iliescu@kth.se} \\
  \And
  Leandro Duarte \\
  KTH Royal Institute of Technology\\
  \texttt{ldr0@kth.se} \\
  \And
  Miguel Arroyo Marquez \\
  KTH Royal Institute of Technology\\
  \texttt{miguela@kth.se} \\
  \And
  Mingyang Chen \\
  KTH Royal Institute of Technology\\
  \texttt{mingyanc@kth.se} \\
}

\begin{document}

\maketitle

\begin{abstract}
  This paper investigates 
\end{abstract}

\section{Introduction}
% What is it that you are trying to solve/achieve and why does it matter

\subsection{Motivation}
Social media platforms have become 

\subsection{Problem Definition}
In this paper, we aim to analyze the 
\begin{itemize}
    \item How do political communities 
\end{itemize}

\subsection{Approach Overview}
We approach this problem through ´

\section{Related Work}
% How does your project relate to previous work

\subsection{Political Polarization in Social Media}
% Discuss previous works on political polarization in social media

\subsection{Community Detection in Temporal Networks}
% Discuss previous works on community detection in temporal networks

\subsection{Reddit as a Platform for Political Discourse}
% Discuss previous works that analyze Reddit for political discourse

\section{Data}
\subsection{Dataset Description}
We use the Reddit Politosphere dataset \cite{Hofmann_Schütze_Pierrehumbert_2022}, which covers more than 600 political subreddits over a 12-year period (2008-2019). The dataset contains:
\begin{itemize}
    \item Network data files 
\end{itemize}

\subsection{Data Preprocessing}
% Describe any preprocessing steps applied to the data

\section{Methodology}
\subsection{Network Representation}
% Describe how the networks are represented

\subsection{Community Detection Algorithms}
We implement multiple community detection algorithms to identify clusters of related subreddits:

\subsubsection{Label Propagation Algorithm}
% Describe the Label Propagation algorithm and how we implement it

\subsubsection{Spectral Clustering}
% Describe the Spectral Clustering algorithm and how we implement it

\subsubsection{Additional Methods (Optional)}
% Describe any additional methods if used (e.g., Louvain method)

\subsection{Temporal Community Tracking}
To track communities over time, we calculate similarity measures between communities in consecutive years:
% Describe the approach for tracking communities over time

\subsection{Event Correlation Analysis}
% Describe how we analyze correlation with political events

\section{Experimental Setup}
\subsection{Implementation Details}
% Describe implementation details, libraries used, etc.

\subsection{Evaluation Metrics}
We evaluate our analysis using several metrics:
\begin{itemize}
    \item Community quality metrics 
\end{itemize}

\section{Results and Discussion}
\subsection{Community Evolution Patterns}
% Present and discuss results on community evolution

\subsection{Correlation with Political Events}
% Present and discuss results on correlation with political events

\subsection{Polarization Trends}
% Present and discuss results on polarization trends

\subsection{Algorithm Comparison}
% Compare the performance of different community detection algorithms

\section{Conclusion}
% Summarize main findings, contributions, and implications

\subsection{Limitations and Future Work}
% Discuss limitations of the study and potential future directions

\bibliographystyle{plain}
\bibliography{references}

\end{document}
